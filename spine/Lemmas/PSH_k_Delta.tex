\section*{Pebble Saturation Hypothesis (PSH$_{k,\Delta}$)}

\begin{theorem}[Pebble Saturation on Bounded-Degree Graphs]
<<<<<<< HEAD
For all $k \ge 2$ and $\Delta \ge 1$, there exists $S=S(k,\Delta,r)$ such that in any graph of maximum degree $\Delta$, any set of vertices that are pairwise equivalent under the r-round $k$-pebble game has size at most $S$.
\end{theorem}

\begin{proof}
On trees, stable $k$-WL equivalence coincides with r-round $k$-pebble equivalence.
Since $k$- in $O_{k,\Delta}(1)$ rounds on trees of bounded degree, the number of stable colors is finite.
=======
For all $k \ge 2$ and $\Delta \ge 1$, there exists $S=N(k,\Delta,r)$ such that in any graph of maximum degree $\Delta$, any set of vertices that are pairwise equivalent under the  $k$-pebble game has size at most $S$.
\end{theorem}

\begin{proof}
Reduce to the , which is a degree-$\Delta$ tree.
On trees, stable $k$-WL equivalence coincides with  $k$-pebble equivalence.
Since $k$- in $O_{k,\Delta}(1)$ rounds on trees of bounded degree, the number of stable colors is finite.
>>>>>>> 9d0a289 (Normalize PSH to PSH_{k,Δ,r} (rank-bounded only))
Thus the number of pebble-equivalence classes is bounded by a constant depending only on $k,\Delta$.
\end{proof}
